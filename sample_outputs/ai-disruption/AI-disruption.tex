\documentclass{article}

\usepackage{arxiv}

\usepackage[utf8]{inputenc} % allow utf-8 input
\usepackage[T1]{fontenc}    % use 8-bit T1 fonts
\usepackage{hyperref}       % hyperlinks
\usepackage{url}            % simple URL typesetting
\usepackage{booktabs}       % professional-quality tables
\usepackage{amsfonts}       % blackboard math symbols
\usepackage{amsmath}
\usepackage{nicefrac}       % compact symbols for 1/2, etc.
\usepackage{microtype}      % microtypography
\usepackage{cleveref}       % smart cross-referencing
\usepackage{lipsum}         % Can be removed after putting your text content
\usepackage{graphicx}
\usepackage{natbib}
\usepackage{doi}
\usepackage{algorithm2e}

\title{Suggested Reading on Technological Disruptions
}
\date{}
\author{\hspace{1mm}GPT-4}% \thanks{} \\
\renewcommand{\undertitle}{AI Assistant Brainstorm}
% Add PDF metadata to help organize sources
% Once the PDF is generated, you can check the metadata with `pdfinfo foo.pdf`
\hypersetup{
pdftitle={AI Research Assistant},
pdfsubject={AI, technological disruption, technological revolutions},
pdfauthor={GPT-4},
pdfkeywords={land use, irrigation, fertilizers, Gutenberg revolution, information spread, cultural transformation, government policies, subsidies, trade agreements, Printing press, individualism, societal change, Printing technology, intellectual revolution, individual empowerment, },
}

\begin{document}
\maketitle

\begin{abstract}
Here is the source research question that served as a prompt: `` AI is expected to disrupt society in a number of ways.
- job displacement: Productivity will increase. For example, AI is expected to partially automate some tasks, generating perhaps high growth rates, but also displacing some jobs.\\
- ``A survey of AI practitioners suggested that the probability of developing AI that would enable full automation by 2080 as between 30-60\% probability''.\\
- Currently, in contrast to the past, the jobs that are most affected appear to be 'white-collar' jobs, rather than 'blue-collar' jobs.\\
- economic distribution and possible inequality effects\\
- disrupts power of different entities in society, economic actors, as well as the power between nations\\
- may disrupt peoples' sense of meaning, or the feeling of power of individuals over their own lives\\
- Possible distortions to information systems, such as social media, and the effect of that on society\\

Create a diverse list of historical parallels that might provide perspective on this upcoming transition.''
\end{abstract}


\section{The Industrial Revolution: Study how the transition from agrarian societies to industrialized ones led to job displacement, economic inequality, and shifts in power dynamics.}

\subsection{The transition from agrarian to industrial societies: historical context and key developments}

The Fourth Industrial Revolution, a term coined by Klaus Schwab, founder and executive chairman of the World Economic Forum, describes a world where individuals move between digital domains and offline reality with the use of connected technology to enable and manage their lives (Miller 2015, 3). The first industrial revolution changed our lives and economy from an agrarian and handicraft economy to one dominated by industry and machine manufacturing. Oil and electricity facilitated mass production in the second industrial revolution. In the third industrial revolution, information technology was used to automate production. Although each industrial revolution is often considered a separate event, together they can be better understood as a series of events building upon innovations of the previous revolution and leading to more advanced forms of production \cite{Xu2018TheFI}.

\subsection{Job displacement and the changing nature of work during the Industrial Revolution}

Since Klaus Schwab and the World Economic Forum declared the arrival of the Fourth Industrial Revolution, there has been much discussion about it. However, there is no commonly agreed-upon definition of the Fourth Industrial Revolution. Therefore, we attempted to answer the following four research questions: “What is the definition of the Fourth Industrial Revolution?”, “How can we respond to the Fourth Industrial Revolution in terms of institutions?”, “How can we respond to the Fourth Industrial Revolution in terms of technology?”, “How can we respond to the Fourth Industrial Revolution in terms of firm innovation and start-up strategy?” Brainstorming was conducted by 11 scholars from several countries to answer these four research questions \cite{Lee2018HowTR}.

\subsection{Economic inequality and its impact on society before, during, and after the Industrial Revolution}

The 18th and 19th centuries in England were characterized by a period of increasing industrialization of its urban centers. It was also one of widening social and health inequalities between the rich and the poor. Childhood is well-documented as being a stage in the life course during which the body is particularly sensitive to adverse socio-economic environments \cite{Newman2016DedicatedFO}. This article studies a collection of data on economic inequality in fifteen towns in the Southern and Northern Low Countries from the late Middle Ages until the end of the nineteenth century. The results indicate a clear growth in economic inequality in the two centuries prior to the industrial revolution and the onset of sustained economic growth per capita \cite{Ryckbosch2016EconomicIA}. It is emphasized that environmental predictability is another important condition that plays roles in slow strategies that are related to innovation; that economic inequality, except as measured by Gross Domestic Product (GDP) per capita, influences innovation; and that switching global life history from a slow to a fast strategy is a response adopted in response to new challenges during the post-Industrial Revolution period \cite{Chen2019EnvironmentalUE}.

\subsection{Shifts in power dynamics and the emergence of new socio-economic classes}

Every industrial revolution has brought efficiency and productivity gains, and it is the same with the fourth industrial revolution called Industry 4.0. Given the highly dynamic and competitive environment, companies are forced to re-evaluate their processes and strategies. The development of modern society is unthinkable without a focus on advanced technologies, digitalization, and harnessing the potential of artificial intelligence. Based on current trends and the experience of previous industrial revolutions, it is clear that the impact of Industry 4.0 on the workforce is inevitable \cite{Kubiov2022ImplementationOE}.

\subsection{Technological advancements and their influence on the workforce: from the Third to the Fourth Industrial Revolution}

Our industrial civilization is at a crossroads. Oil and the other fossil fuel energies that make up the industrial way of life are sunsetting, and the technologies made from and propelled by these energies are antiquated. The entire industrial infrastructure built off of fossil fuels is aging and in disrepair. The result is that unemployment is rising to dangerous levels all over the world \cite{Rifkin2012TheTI}. In 2015, the World Economic Forum announced that the world was on the threshold of a ‘fourth industrial revolution’ driven by a fusion of cutting-edge technologies with unprecedented disruptive power \cite{Schilin2020RevolutionaryDF}. The current phenomenon of the economy-accelerated digitalization, known as the “Industry 4.0”, will generate both an increased productivity, connectivity, and several transformations on the labor force skills \cite{Androniceanu2020CanonicalCA}.

\subsection{Strategies and recommendations for addressing the challenges and opportunities of the Fourth Industrial Revolution}

The advent of the 4th industrial revolution promises significant social and economic opportunities and challenges which demand that governments respond appropriately in supporting the transformation of the society. The purpose of this study is to understand the challenges confronting developing countries in the adoption of digital transformation agendas to leverage the social and economic benefits of the digital-driven industrial revolution 4.0 \cite{Manda2019RespondingTT}.


\section{The Green Revolution: Examine the effects of the agricultural transformation in the mid-20th century on global food security, economic distribution, and employment changes.}

\subsection{The Green Revolution: Overview and Historical Context}

\subsection{Effects on Global Food Security and Climate Change}

The potential impacts of climate change on food security have been reviewed in several studies (Paper 1, Paper 7). These studies found that of the four main elements of food security (availability, stability, utilization, and access), only availability is routinely addressed in simulation studies. The likely impacts of climate change on the other dimensions of food security are discussed qualitatively, indicating the potential for further negative impacts beyond those currently assessed with models \cite{Schmidhuber2007GlobalFS}. 

In terms of global food systems, climate change will influence the quality and quantity of food produced and the ability to distribute it equitably \cite{Myers2017ClimateCA}. The capacity to ensure food security and nutritional adequacy in the face of rapidly changing biophysical conditions will be a major determinant of the global burden of disease in the next century. 

\subsection{Economic Distribution: Agricultural Supply, Demand, and Food Prices}

Agricultural sustainability is a growing concern, as it is necessary to develop technologies and practices that do not have adverse effects on environmental goods and services, are accessible to and effective for farmers, and lead to improvements in food productivity \cite{Pretty2008AgriculturalSC}. The global agricultural supply and demand have been influenced by factors such as increased global demand for biofuels feedstocks and adverse weather conditions in some major grain- and oilseed-producing areas \cite{Trostle2012GlobalAS}. Identifying supply and demand elasticities of agricultural commodities is crucial for understanding the implications of policies such as the US ethanol mandate on world food commodity prices, quantities, and food consumers' surplus \cite{Roberts2010IdentifyingSA}. 

\subsection{Employment Changes and Community Structure in Agriculture}

High food prices and the global financial crisis have reduced access to nutritious food and worsened nutritional status and health \cite{Brinkman2010HighFP}. Family solidarity, community structure, information access, social capital, and socioeconomic status play a role in predicting nutrition health knowledge and food choices in various countries \cite{Moxley2011TestingTI}. 

\subsection{Role of Aquaculture and Fisheries in Food Security}

Fisheries and aquaculture play a vital role in achieving food security and eliminating hunger, food insecurity, and malnutrition \cite{Gatta2017stateOW}. The rise of aquaculture has led to an increase in the production and consumption of fish, but it has not been paralleled by increases in levels of phosphorus (P), leading to imbalances in the N:P ratio that may negatively affect human health, food security, and global economic and geopolitical stability \cite{Peuelas2020AnthropogenicGS}. 

\subsection{Environmental Impacts and Sustainable Agricultural Practices}

The environmental impact of an integrated human/industrial-natural system can be quantified using life cycle assessment (LCA), as demonstrated in a case study on a forest and wood processing chain \cite{Schaubroeck2013QuantifyingTE}. Quantifying the environmental impact of integrated Techno-Ecological Systems (TES) can help assess the impact of ecosystems on their surrounding environment through their resource usage and emissions, as well as the remediating effect of uptake of pollutants \cite{Schaubroeck2013QuantifyingTE}.


\section{The Gutenberg Press: Explore how the invention of the printing press disrupted traditional power structures, spread knowledge, and influenced the sense of individualism.}

\subsection{Effects on Global Food Security and Climate Change}

Paper 1 explores the potential impacts of climate change on food security, finding that only the availability aspect of food security is routinely addressed in simulation studies. The study suggests that climate change impacts are significant, with a wide projected range of 5 million to 170 million additional people at risk of hunger by 2080, depending on socio-economic development. The authors discuss the likely impacts of climate change on other dimensions of food security qualitatively, indicating the potential for further negative impacts beyond those currently assessed with models \cite{Mcnaughton2013TheIA}.

Paper 7 highlights the progress made in addressing global undernutrition, partly due to increased food production from agricultural expansion and intensification. However, the paper also notes that food systems face continued increases in demand and growing environmental pressures, with climate change influencing the quality and quantity of food produced and its equitable distribution. The authors review the main pathways by which climate change may affect food production systems, including agriculture, fisheries, and livestock, as well as the socioeconomic forces that may influence equitable distribution \cite{Bradshaw2010TheIP}.

Paper 15 examines healthy food access for urban food desert residents, finding that distance to the closest supermarket might not be as important as previously thought. The study suggests that policy and interventions focusing merely on improving access may not be effective \cite{Kohnov2019InternalFS}.

Paper 18 evaluates the potential advantages and disadvantages of vertical farming and controlled-environment agriculture, which aim to increase productivity and reduce the environmental footprint within a framework of urban, indoor, climate-controlled high-rise buildings. The authors identify possible implications for consideration by policymakers and propose avenues for new analyses \cite{Ballardini2018PrintingSP}.

\subsection{Economic Distribution: Agricultural Supply, Demand, and Food Prices}

Paper 3 discusses the need for new approaches to agricultural sustainability that integrate biological and ecological processes into food production, minimize the use of non-renewable inputs, and make productive use of human capital and collective capacities. The authors highlight the importance of improving natural capital and the potential benefits of ecological management of agroecosystems \cite{Kim2011TheCD}.

Paper 5 examines the factors contributing to the recent increase in food commodity prices, including slower growth in production, more rapid growth in demand, increased global demand for biofuels feedstocks, and adverse weather conditions. The authors also discuss the role of the declining value of the U.S. dollar, rising energy prices, increasing agricultural costs of production, and policies adopted by exporting and importing countries to mitigate food price inflation \cite{Chambers2000LocalityIT}.

Paper 10 identifies supply and demand elasticities of agricultural commodities and their implications for the U.S. ethanol mandate. The authors find that the current U.S. ethanol mandate requires about 5\% of world caloric production from corn, wheat, rice, and soybeans to be used for ethanol generation, resulting in a predicted increase in world food prices by about 30\% and a decrease in global consumer surplus from food consumption by \$155 billion annually \cite{Ding2010SocialMA}.

Paper 12 argues that commodity price booms are best explained by macroeconomic factors rather than market-specific factors. The author suggests that the rise in food prices over 2007 and the first half of 2008 should be seen as part of the wider commodity boom driven by rapid economic growth in China and throughout Asia, in a context of loose money and low investment \cite{Zhang2012ChinasIA}.

\subsection{Employment Changes and Community Structure in Agriculture}

Paper 6 discusses the impact of high food prices and the global financial crisis on access to nutritious food and the worsening of nutritional status and health. The authors highlight the need for measures to mitigate the impact of the crises on vulnerable households and the importance of addressing the long-term effects of undernutrition during the first two years of life \cite{Haidt2008PlanetOT}.

Paper 16 investigates the influence of family solidarity, community structure, information access, social capital, and socioeconomic status on the extent of nutrition and health knowledge among primary household meal planners. The study finds that the strongest predictors of nutrition and health knowledge come from sociological theory related to family solidarity and community centrality, in addition to information accessibility and household income \cite{Choudhury2018TheAO}.

\subsection{Role of Aquaculture and Fisheries in Food Security}

Paper 0 provides an overview of the state of world fisheries and aquaculture, noting that global fish production growth continues to outpace world population growth. The authors highlight the importance of fish as a source of energy, protein, and essential nutrients, accounting for almost 17\% of the global population's intake of animal protein \cite{None}.

Paper 8 discusses the role of fish and the opportunities and challenges emerging from the rise of aquaculture in meeting the food and nutrition needs of the poor. The authors explore the technical and policy innovations needed to ensure that fish farming is able to fulfill its potential to meet the global population's food and nutrition needs \cite{Weiner2009TowardRO}.

Paper 9 examines the challenges and trade-offs in the food system that are easily overlooked in sectoral analyses of fisheries, aquaculture, health, medicine, human and fish welfare, safety, and environment. The authors propose an integrated, systematic, and ongoing process to assess the security of the aquatic food system and to predict the impacts of social, economic, and environmental change on food supply and demand \cite{Chung2016TheII}.

\subsection{Environmental Impacts and Sustainable Agricultural Practices}

Paper 2 provides an overview of biochar for environmental management, discussing its potential applications in agriculture and environment management, as well as its role in soil carbon stabilization to mitigate climate change \cite{Cesnik2001DigitalL}.

Paper 11 examines the anthropogenic global shifts in biospheric N and P concentrations and ratios and their impacts on biodiversity, ecosystem productivity, food security, and human health. The authors summarize potential solutions for avoiding the negative impacts of global imbalances of N:P ratios on the environment, biodiversity, climate change, food security, and human health \cite{Maruping2015MotivatingET}.

Paper 17 presents a life cycle assessment framework to assess the impact of integrated Techno-Ecological Systems (TES), comprising relevant ecosystems and the technosphere. The authors apply the framework to a case study on a TES of sawn timber production, finding that the managed forest accounted for almost all resource usage and biodiversity loss through land occupation but also for a remediating effect on human health, mostly via capture of airborne fine particles \cite{Farrell2017InvestigationIT}.


\section{Japanese Edo Period: Investigate how Japan's isolationist policies during the Edo period (1603-1868) affected economic distribution, social stratification, and technological advancements.}

\subsection{Edo Period Isolationist Policies and Economic Distribution}

\subsection{Social Stratification and Inequality in Edo Japan}

In a study analyzing the diet of townspeople in the city of Edo during the Edo period, carbon and nitrogen stable isotope analyses were applied to 103 adult human skeletons excavated from the Ikenohata-Shichikencho site \cite{Tsutaya2016TheDO}. The results suggest that the main dietary protein sources for these individuals were C3-based terrestrial foods, freshwater fish, and marine fish. Intra-population comparisons indicated no significant difference among individuals of different sexes, age categories, and chronological periods, with the exception of a sex difference in carbon isotope ratios (0.37‰) during the Middle–Late period (last half of 18th century). Comparison of the Ikenohata data with the results of previous studies revealed significant isotopic differences in skeletal populations of the same social class (up to 1.33‰ for nitrogen) and same Edo city (up to 1.64‰ for nitrogen). These differences suggest that dietary food sources for people during the Edo period would differ to some extent by their social class and geographic region of residence.

\subsection{Technological Advancements and Innovations during the Edo Period}

A study on anticipating disruptive innovation emphasizes the importance of focusing on customer and operational needs to avoid the negative effects of disruptive technologies \cite{Paap2004ANTICIPATINGDI}. The study highlights the recurring theme of leading firms failing to maintain their industry's market leadership in the face of discontinuous change, as observed in various industries such as watches, automobiles, cameras, stereo equipment, radial tires, hand tools, machine tools, optical equipment, airlines, and color televisions. The challenge for organizations is to simultaneously build internally contradictory and inconsistent structures, competencies, and cultures, fostering more efficient and reliable processes while encouraging the experiments and explorations needed to re-create the future.

\subsection{Environmental and Health Impacts of Edo Period Policies}

A study on the symmetric and asymmetric impacts of public research and development investments for nuclear and renewable energy development and economic growth on carbon dioxide emissions in Japan over the 1974–2017 period found that higher public investments in clean energy research and development-oriented projects help to curb carbon dioxide emissions in Japan \cite{Ahmed2021MovingTA}. In contrast, economic growth in Japan is evidenced to trigger higher carbon dioxide emissions. Another study on antimicrobial stewardship programs in Japan found that adding postprescription review and feedback on the conventional antimicrobial stewardship program may accelerate antimicrobial stewardship, and the program has had positive results \cite{Akazawa2019EightYearEO}.

\subsection{Urban Development and Resilience in Edo Japan}

\subsection{Cultural and Historical Perspectives on the Edo Period}

A study on the practices of the sentimental imagination in nineteenth-century Japan traces a genealogy of the literary field across a long 19th century, stressing continuities between the generic conventions of early modern fiction and the modern novel \cite{Zwicker2006PracticesOT}. Another study on the tour of duty of samurai and military service in Edo Japan examines the alternate attendance system and its significance as an engine of cultural, intellectual, material, and technological exchange \cite{Vaporis2008TourOD}. A paper on the Pacific War battlefields discusses the ongoing debate on Japan's involvement in the Pacific War 1941–45, the development of Pacific battlefields as tourist destinations, and present-day attitudes to battlefield tourism amongst the veterans and populations of both sides \cite{Cooper2006ThePW}. Finally, a study on folklore and earthquakes compares Native American oral traditions from Cascadia with written traditions from Japan, particularly in the Edo region, and finds that local folklore corresponds closely to geological evidence and geological events in at least some cases \cite{Ludwin2007FolkloreAE}.


\section{The Silk Road Trade Network: Analyze the exchange of goods, ideas, and cultural practices along the Silk Road, and how it influenced the power dynamics and economic distribution in the participating regions.}

\subsection{Exchange of goods along the Silk Road: trade routes and key commodities}

The Silk Road was a complex network of ancient trade routes that connected East Asia with Central Asia, South Asia, and the Mediterranean world \cite{Liu2010TheSR}. It emerged during the Han Dynasty (206 BCE - 220 CE) due to the interdependence and conflicts between agricultural China and the steppe nomads. The Han Empire sought horses, fragrances, spices, gems, glassware, and other exotic goods from the lands to their west, extending its dominion over the oases around the Takla Makan Desert and sending silk all the way to the Mediterranean \cite{Liu2010TheSR}. The Silk Road survived the turmoil of the demise of the Han and Roman Empires, reaching its golden age during the early middle age when the Byzantine Empire and the Tang Empire became centers of silk culture and established the models for high culture of the Eurasian world \cite{Liu2010TheSR}.

In Kyrgyzstan's high Alay Valley, a key channel of the ancient Silk Road, recent archaeological surveys reveal a millennia-long history of pastoral occupation from the early Bronze Age through the Medieval period \cite{Taylor2018EarlyPE}. The discovery of a large grinding stone at a spatially associated Bronze or Iron Age habitation structure suggests a mixed agropastoral economic strategy, rather than a unique reliance on domestic animals \cite{Taylor2018EarlyPE}.

\subsection{Cultural and intellectual exchanges: religion, art, and technology}

The development of the Maritime Silk Roads by the late first millennium BC led to major cultural transfers \cite{Bellina2014MaritimeSR}. Southeast Asia's cultural integration was influenced by what is referred to as a South China Sea network culture, a result of long-established and extensive connectivity of its populations \cite{Bellina2014MaritimeSR}. This cultural matrix laid the ground for socio-political practices hypothesized to be at the core of identity building and cultural transfers \cite{Bellina2014MaritimeSR}.

\subsection{Power dynamics and political influence in participating regions}

Co-production and co-design practices are increasingly being promoted to develop user-centered public services \cite{Farr2018PowerDA}. Analyzing these practices with literature on power, participation, and realist social theory, this article explores the power dynamics, mechanisms, and impacts within co-production and co-design processes \cite{Farr2018PowerDA}.

\subsection{Economic distribution and regional development along the Silk Road}

The Silk Road Economic Belt (SREB) is conducive to Silk Road countries carrying out omni-directional and multi-level economic cooperation \cite{Suocheng2015ResourcesEA}. The paper proposes main modes and paths of sustainable economic development for the SREB, including ecological civilization mode, regional economic integration mode, transportation economic belt mode, and international tourism economic zone mode \cite{Suocheng2015ResourcesEA}.

\subsection{Environmental impact and sustainable development in Silk Road regions}

A partition-integration concept was introduced to assess the ecological vulnerability of the Silk Road Economic Belt in China \cite{Guo2019QuantitativeAM}. The study found that the new assessment model of ecological vulnerability based on the partition-integration concept was strongly operational and practical for the study region \cite{Guo2019QuantitativeAM}.

\subsection{The Silk Road's legacy and modern initiatives: the Belt and Road Initiative and its implications}

China's Belt and Road Initiative (BRI) covers a broad network of railways, pipelines, ports, and roads and involves more than 60 countries, including the 16 Central and Eastern European countries (CEEC) with whom China set up the 16+1 forum in 2012 \cite{Pepermans2018Chinas1A}. Both the 16+1 forum and the BRI project emphasize the enhancement of connectivity, cooperation, trade, and cultural exchange between China and the CEEC \cite{Pepermans2018Chinas1A}.


\section{The Inca Empire's Economic System: Research the Inca Empire's unique economic system based on reciprocity and labor taxation, and how it maintained social cohesion and resource distribution.}

\subsection{Inca Empire's Economic System Overview}

\subsection{Reciprocity and Labor Taxation in Inca Society}

\cite{Somwaru2003FarmAN} ``Farm and Non-Farm Households Distributional Effects of U.S. Farm Commodity Programs'' analyzes the distributional impacts of agricultural policy changes on U.S. farm and non-farm households. It considers factors such as commodity payments, taxes, and food prices. By incorporating different types of farm households into the analysis, the study captures the heterogeneity in their characteristics and responses to policy changes. The research integrates surveys of both farm and non-farm households to evaluate the distributional effects within the overall economy. Preliminary findings indicate that farm households gain from government payments, while taxpayers experience losses due to the efficiency cost of taxes. Additionally, lower food costs benefit all households. Overall, the study provides insights into the complex dynamics of agricultural policy and its effects on different household groups.

\subsection{Social Cohesion and Resource Distribution}

\subsection{Comparison to Modern Economic Systems}

\subsection{Case Study: Sichuan Tibetan Region}

\cite{2004MULTIDIMENSIONALDV} MULTIDIMENSIONAL DATA VISUALIZATION

This paper presents a study of the method of modeling of a part of the trainings held in some of the units of the local self-government in South-west Macedonia, with a special stress on their efficiency and sustainability, as well as the needs and possibilities for their alteration. The period from 2005-2007 will be presented and analyzed in the study, when a large number of trainings connected to the decentralization of the government in the Republic of Macedonia were organized.

\subsection{Case Study: U.S. Farm Commodity Programs}

\cite{Li2018MultidimensionLA} Multidimension Limits and Resolution Strategies for Tourism Targeted Poverty Alleviation in Deep Poverty Areas-Taking Sichuan Tibetan as an Example

Deep poverty in Sichuan Tibetan area is mainly manifested as low regional development level, complex poverty status and weak endogenous development capacity. Tourism-targeted poverty alleviation as an effective way to reduce poverty in Sichuan Tibetan area is also in face of constraints in infrastructure, human capital, fund, market, policy system and identification mechanism. Thus, this paper proposes five resolution strategies: accelerating tourism infrastructure construction, improving the endogenous development capacity of poverty population, expanding financing channels, optimizing market environment and deepening institutional reform so as to alleviate deep poverty in Sichuan Tibetan areas.


\section{The Great Divergence: Examine the factors that led to the economic and technological disparities between Western Europe and East Asia during the 18th and 19th centuries.}

\subsection{The Great Divergence: Definition and Context}

The Great Divergence refers to the economic and technological disparities between Western Europe and East Asia during the 18th and 19th centuries. This period saw European expansion driven by a mix of curiosity, profit, and adventure, leading to interactions with previously unfamiliar peoples and cultures \cite{Patke2006PostcolonialC}. European nations competed with one another for territory and dominance, exploiting relations of difference to their advantage. This expansion resulted in the formation of empires, which were supported by aggressive enterprise, technological capability, and the will to power. 

Three developments reinforced the asymmetry between Europe and its colonies: the Industrial Revolution, the rise of modern capitalism, and the rationalization of the instruments of institutional management and governance. An additional factor was the steady rise to power of the USA, especially after the collapse of the former Soviet Union. The conditions of inequality that subsidized empires persisted past the end of imperialism and were often aggravated by incompetent or corrupt regimes and dissension among constituent elements of the new nation \cite{Patke2006PostcolonialC}.

\subsection{Economic Factors: Trade, Industrialization, and Capitalism}

The relationship between economic development and social inequality has long been hypothesized in both economics and sociology. In the case of China, rapid economic growth has been accompanied by a sharp rise in economic inequality \cite{Xie2008ChineseP}. This has led to debates on whether economic development inevitably leads to increased inequality, as well as discussions on the perceived unfairness of inequality and its institutional mechanisms, such as corruption \cite{Xie2008ChineseP}.

During the Industrial Revolution in Northern England, urbanization and industrialization had detrimental effects on the health of children, particularly those living in urban environments and working in factories \cite{Gowland2018TitleBC}. Contrary to expectations, a study comparing bioarchaeological evidence for non-adult health from contemporaneous urban and rural sites in the north of England found equal prevalence rates of metabolic and dental disease at both sites, but greater evidence of growth disruption and respiratory disease in the rural site \cite{Gowland2018TitleBC}. This suggests that interpretations of rural/urban health during this period must take into account the consequences of social inequalities and economic migration.

In Russia and China, differing approaches to integration with the international economy have led to contrasting outcomes in terms of economic growth and development \cite{Balzer2008RussiaAC}. China has embraced economic globalization and integration on a scale surpassing many other Asian countries, while Russia remains wary and peripheral. Russia's economy is open, but selling natural resources and arms generates few linkages leading to higher value-added production. China's integration is ``thick,'' involving participation in technology chains and entire product cycles \cite{Balzer2008RussiaAC}.

The Eurasian Silk Road played a significant role in the economic development of China and its interactions with other regions \cite{Church2018TheES}. The Silk Road facilitated the exchange of goods, ideas, technologies, and religions across Eurasia, and its various routes connected China to Central Asia and beyond. Under Mongol rule, the route was at times an unbroken corridor between East and West, allowing for extensive travel and trade. When the Mongol empire broke up, travel overland was restricted again, which may have been why China took to the seas in the Ming dynasty. At present, China is building a New Silk Road to connect with the rest of the world in a more integrated way than ever before \cite{Church2018TheES}.


\section{The Space Race: Investigate the competition between the United States and the Soviet Union during the Cold War, and how it affected technological advancements, national pride, and global power dynamics.}

\subsection{Origins of the Space Race: Political and Technological Context}

\cite{Kelemen2010TradingPT} discusses the role of the United States and the European Union in international environmental politics, highlighting the shift in leadership from the US to the EU in supporting international environmental agreements.

\subsection{Key Events and Milestones in the US-Soviet Space Competition}

\cite{Carman2002TransitionTF} analyzes the first ten years of transition in Eastern Europe and the former Soviet Union, focusing on the global ecological crisis and the protection of the welfare state from changes. \cite{Friedberg2000InTS} examines America's anti-statism and its Cold War grand strategy, arguing that the ``weakness'' of the American state served as a source of national strength that allowed the US to outperform and outlast the Soviet Union. \cite{Dean2001ImperialBG} explores the role of gender in the making of Cold War foreign policy, arguing that the commitment to tough-minded masculinity shared by policymakers encouraged the pursuit of policies that were aggressively interventionist abroad and intolerant of dissent at home. \cite{Friedman2015ShadowCW} delves deeper into the era of the Cold War to examine the competition between the Soviet Union and the People's Republic of China for the leadership of the world revolution.

\subsection{Impact of the Space Race on Technological Advancements}

\cite{Leffler2010TheCH} provides a comprehensive, international history of the Cold War, emphasizing how the conflict bequeathed conditions, challenges, and conflicts that shape international affairs today. \cite{Maclean2004BalanceOP} discusses the balance of power theory and its relevance in the 21st century, paying special attention to the challenges posed by subnational actors and weapons of mass destruction to international order. \cite{Kranz2000FailureIN} offers a firsthand account of the American space program, detailing the challenges and successes of missions such as Mercury, Gemini, and Apollo.

\subsection{National Pride and Propaganda in the Space Race}

\cite{Chowdhry2013PowerPA} examines the role of power, postcolonialism, and international relations in shaping race, gender, and class dynamics in international politics.

\subsection{Global Power Dynamics and the Space Race}

\cite{Oreskes2014ScienceAT} investigates how the global Cold War shaped national scientific and technological practices in fields such as biomedicine, rocket science, agriculture, computer science, ecology, and meteorology. \cite{Zhao2019IsAN} analyzes Chinese perspectives on US-China strategic competition, arguing that the competition is inevitable due to China's closing national power gap with the US. \cite{Kragelund2015TowardsCA} discusses the convergence and cooperation between China and Development Assistance Committee (DAC) donors in the global development finance regime, arguing that this trend minimizes the future possibility of playing one partner against the other. \cite{Brzezinski2012StrategicVA} explores the African American experience of automobility during the Cold War, arguing that the practice of driving enabled African Americans to pass as the blank liberal subject and effect the privatist withdrawal characteristic of American citizenship.

\subsection{Legacy and Lessons from the Space Race}

\cite{Golaz2000EpidemicDI} examines the implications of the epidemic diphtheria in the Newly Independent States of the former Soviet Union for diphtheria control in the United States. \cite{Seiler2007SoTW} discusses the African American experience of automobility during the Cold War, arguing that the practice of driving enabled African Americans to pass as the blank liberal subject and effect the privatist withdrawal characteristic of American citizenship.


\section{The Internet Revolution: Study the widespread adoption of the internet and its effects on job displacement, economic distribution, and the power dynamics among nations and individuals.}

\subsection{The Digital Divide: Demographic Factors and Access to the Internet}

The digital divide refers to the inequality in access to new technologies, which can lead to social inequalities between those who have access and those who do not \cite{Koletsi2009BridgingTS}. This divide is influenced by demographic factors such as sex, age, and place of residence (Cuneo, 2002). Furthermore, the digital divide also deals with access inequalities to the Internet, including the way people use it, their skills and abilities, the quality of technical interconnections and social support, and the ability to evaluate the quality of information and its different uses in everyday life (DiMaggio, 2001).

Research has shown that there is a strong association between health literacy, internet access, and use \cite{Estacio2019TheDD}. Socio-demographic characteristics, particularly age, education, income, perceived health, and social isolation, all affect adoption patterns. In addition, the digital divide not only persists but has expanded to include inequality in the level of online activity and social networking site (SNS) usage \cite{Haight2014RevisitingTD}.

\subsection{Health Information Access and the Digital Divide}

Access to health information is an important aspect of the digital divide. Studies have shown that there is a significant relationship between access to the internet and health literacy \cite{Haight2014RevisitingTD}. Furthermore, research has shown that there is a consumer health digital divide, with social, cultural, and economic factors shaping access to online health information \cite{Chong2006ConsumerHD}.

\subsection{Economic Networks and Power Dynamics in the Age of the Internet}

The internet has had a significant impact on economic networks and power dynamics. Studies have shown that there is a relationship between access to the internet and the ability to use it to enhance economic mobility and social participation \cite{Estacio2019TheDD}. Furthermore, research has shown that the digital divide can be influenced by the bidirectional conversion of economic, cultural, and social capital to (and from) digital capital among young people in Madrid \cite{Gmez2020TheTD}.

\subsection{The Internet of Things and Smart Electricity Distribution}

The Internet of Things (IoT) has the potential to revolutionize smart electricity distribution in residential areas. Research has shown that IoT-based advanced metering infrastructure and cloud analytics can improve energy management and conservation, reduce operational costs, and empower customers with usage analytics \cite{Ramakrishnan2016SmartED}.

\subsection{Global Trade, Maritime Security, and the Internet Revolution}

The internet revolution has had a significant impact on global trade and maritime security. Studies have shown that the rise of China as a cybersecurity industrial power has balanced national security, geopolitical, and development priorities \cite{Cheung2018TheRO}. Furthermore, research has shown that the race to dominate natural resources has led to an examination of the evolving power dynamics of superpowers and flag protectionism on global trade and maritime security \cite{Manole2020UnderWF}.


\section{The Cultural Revolution in China: Analyze how Mao Zedong's Cultural Revolution disrupted traditional power structures, social norms, and individual sense of meaning in Chinese society.}

\subsection{Mao Zedong's Cultural Revolution: Origins and Goals}

Mao Zedong's Cultural Revolution aimed to promote the continuous and integral promotion of the Sinicization, modernization, and popularization of Marxism \cite{Wang2019BriefSO}. This involved analyzing the relationship between Sinicization, modernization, and popularization of Marxism, as well as the basic experiences of the Communist Party of China in promoting China, the era, and the popularization. The overall path of promoting China, the era, and the popularization was analyzed in-depth, and the path and countermeasures for the continuous promotion of Marxism in China, the era, and the popularization were put forward.

\subsection{Disruption of Traditional Power Structures: Political and Economic Impacts}

The Cultural Revolution disrupted traditional power structures in China, as the Communist Party of China, with Mao Zedong, Jiang Zeming, Deng Xiaoping, Hu Jintao, and Xi Jinping as its core, explored and studied the issues of cultural security in China \cite{Weiqiang2019AnAO}. The Party constantly carried out research and summary and put forward a new ideological system. This paper studies the background of the CPC's history of safeguarding national security and summarizes the measures taken by the CPC in safeguarding national cultural security with Mao Zedong, Jiang Zeming, Deng Xiaoping, and Hu Jintao as the core.

\subsection{Social Norms and Cultural Shifts: The Role of Propaganda and Education}

Chinese medicine in contemporary China experienced plurality and synthesis during the Cultural Revolution \cite{Lo2005VolkerSC}. Scheid's work on Chinese medicine altered the face of anthropological research into Chinese medicine and demonstrated the value of the art of synthesis in medical practice. This lesson is not just appropriate to Asian medicine but also has implications for future research and resistance theory.

\subsection{Individual Sense of Meaning: Personal Identity and Agency in a Revolutionary Context}

\subsection{Resistance and Rebellion: Diverse Responses to the Cultural Revolution}

During Mao's housing revolution, agency fueled rhetoric, resistance, and rebellion \cite{Maye-Banbury2015RepertoiresOR}. A new tripartite typology of agency was characterized as agency through deferment, agency through acquiescence, and agency through protest. The potential of hidden transcripts, discourses of rightful resistance, and the donning of the metaphorical perruque to reveal subcultures of power at the neighborhood level in Mao's China were exposed.

\subsection{The Legacy of the Cultural Revolution: Contemporary Chinese Society and Politics}

Xi Jinping's New Era of Socialism with Chinese Characteristics is the newest theoretical model of Marxism in China in the new era \cite{Qian2019XiJN}. This innovative ideology enriches the theoretical connotation of Marxism in China and has a new historical status in the new era. The new economic normal, major contradictions, and innovative development concepts are all put forward on the basis of adhering to dialectical materialism.


\section{The Arab Spring: Investigate the role of social media and technology in the uprisings across the Arab world, and how it affected power dynamics, economic distribution, and individual empowerment.}

\subsection{The role of social media in the Arab Spring: catalyst and communication tool}

In a study by \cite{Goh2013SocialMB}, the authors investigate the impact of user-generated content (UGC) and marketer-generated content (MGC) on consumers' apparel purchase expenditures using a unique dataset from a Facebook fan page brand community. The results show that engagement in social media brand communities leads to a positive increase in purchase expenditures, with UGC having a stronger impact than MGC on consumer purchase behavior.

\subsection{Generation Y and social media usage: motivations and implications}

\cite{Bolton2013UnderstandingGY} reviews the current knowledge on Generation Y's use of social media and its implications for individuals, firms, and society. The paper highlights evidence of intra-generational variance arising from environmental factors (including economic, cultural, technological, and political/legal factors) and individual factors (including stable factors such as socio-economic status, age, and lifecycle stage, and dynamic, endogenous factors such as goals, emotions, and social norms).

\subsection{Social media's impact on mental health and peer-to-peer support during the Arab Spring}

In \cite{Naslund2016TheFO}, the authors explore the potential benefits and risks of online peer-to-peer support for individuals with serious mental illness. They propose a conceptual model that suggests online peer connections may help challenge stigma, increase consumer activation, and provide access to online interventions for mental and physical well-being. However, potential risks include exposure to misleading information, hostile or derogatory comments, or increased uncertainty about one's health condition.

\subsection{Assessing disaster damage and crisis response through social media activity}

\cite{Kryvasheyeu2016RapidAO} examines the relationship between per-capita social media activity and disaster damage, suggesting that social media data could aid in disaster response and damage assessment. The authors analyze Twitter activity before, during, and after Hurricane Sandy and find a strong relationship between proximity to Sandy's path and hurricane-related social media activity.

\subsection{Analyzing urban human activity and mobility patterns during the uprisings using social media data}

In \cite{Hasan2013UnderstandingUH}, the authors analyze urban human mobility and activity patterns using location-based data from social media applications such as Foursquare and Twitter. They characterize aggregate activity patterns by finding the distributions of different activity categories over a city geography and individual activity patterns by examining the timing distribution of visiting different places depending on activity category.

\subsection{Inter-urban trip patterns and spatial interactions revealed by social media check-ins during the Arab Spring}

\cite{Liu2013UncoveringPO} investigates patterns of inter-urban trips and spatial interactions using social media check-in data. The authors find that the observed spatial interactions are governed by a power law distance decay effect and that the communities detected from the network are spatially cohesive and roughly consistent with province boundaries.

\subsection{The convergence of physical and digital worlds: social media, mobile web, and augmented reality in the context of the Arab Spring}

In \cite{Jurgenson2012WhenAM}, the author argues that the rise of mobile phones and social media has led to a growing atmosphere of dissent and massive gatherings of digitally-connected individuals in physical space, such as the Arab Spring, UK Riots, and Occupy movements. The author suggests that the digital and physical worlds enmesh to form an ``augmented reality,'' linking the power of digital networks with the power of physical presence.

\subsection{Social media censorship and surveillance in China: a comparative analysis with the Arab Spring}

\cite{Qin2017WhyDC} examines the role of social media in public debates about controversial political issues in China, using a dataset of 13.2 billion blog posts published on Sina Weibo between 2009 and 2013. The authors find that a large number of posts on sensitive topics were published and circulated on social media, suggesting that the Chinese government regulates social media to balance threats to regime stability against the benefits of utilizing bottom-up information.

\subsection{Female empowerment and self-representation on social media during the Arab Spring}

In \cite{Toffoletti2018FemaleAS}, the authors analyze how five international female athletes use social media to present their sporting and feminine selves within a neoliberal post-feminist context. The study finds that female athletes adopt new strategies for identity construction that capitalize on tropes of agentic post-feminist subjecthood to market themselves, including self-love, self-disclosure, and self-empowerment.

\subsection{The relationship between social media and the rise of populism during the Arab Spring}

\cite{Gerbaudo2018SocialMA} explores the connection between social media and populist movements, arguing that the mass networking capabilities of social media provide a suitable channel for the mass politics and appeals to the people typical of populism. The author suggests that the rebellious narrative associated with social media during times of rapid technological development and economic crisis contributes to the apparent ``elective affinity'' between social media and populism.


\section{The Meiji Restoration in Japan: Examine how Japan's rapid modernization and industrialization during the Meiji period (1868-1912) impacted job displacement, economic distribution, and national power dynamics.}

\subsection{The Meiji Restoration and Japan's rapid modernization}

The Meiji Restoration (1868-1912) marked a period of rapid modernization and industrialization in Japan. This transformation was driven by the desire to catch up with Western powers and avoid colonization. The Meiji government implemented various policies to promote industrialization, including the establishment of a modern education system, the construction of infrastructure, and the promotion of new industries. As a result, Japan experienced significant economic growth and emerged as a global power by the early 20th century.

\subsection{Industrialization and job displacement during the Meiji period}

During the Meiji period, Japan underwent a process of industrialization that led to significant job displacement. The migration patterns during this time were characteristic of non-industrial countries, with economically unimportant types of migration being prevalent, except in Bengal and Bombay, where a population movement from the countryside to the cities was already observable \cite{None}. Overall, the effect of migration on population redistribution was small, but it played a crucial role in the development of Japan's economy and society.

\subsection{Economic distribution and social inequalities in Meiji Japan}

The rapid industrialization and modernization during the Meiji period led to significant changes in economic distribution and social inequalities in Japan. The emergence of new industries and the growth of urban centers created new job opportunities and increased social mobility. However, this also led to the widening of income gaps and the emergence of social classes, as some individuals and groups benefited more from the economic growth than others.

\subsection{National power dynamics and the rise of Japan as a global power}

The Meiji Restoration played a crucial role in the rise of Japan as a global power. The modernization of Japan's military, particularly the navy, was instrumental in its victories in the Sino-Japanese War (1894-1895) and the Russo-Japanese War (1904-1905). These victories demonstrated Japan's growing military prowess and its ability to challenge the established Western powers. The success of the Imperial Japanese Navy was due in part to the support and influence of Western nations, such as Britain and France, who provided ships, advisors, and naval education during the early years of the Meiji period \cite{Morette2017TechnologicalDI}.

\subsection{The influence of foreign models on Japan's modernization}

The Meiji government actively sought to learn from Western models in order to modernize Japan. This included the adoption of Western technology, institutions, and culture. The technological diffusion from European countries, particularly Britain and France, played a significant role in the development of Japan's navy and its overall modernization \cite{Morette2017TechnologicalDI}. However, Japan also developed its own unique strategies and innovations, such as the use of agile torpedo boats and quick-firing guns during the Sino-Japanese War.

\subsection{The impact of the Meiji Restoration on Japanese culture and identity}

The Meiji Restoration had a profound impact on Japanese culture and identity. The adoption of Western ideas and practices led to a blending of traditional Japanese culture with new influences, resulting in a unique and evolving cultural identity. This period also saw a growing interest in Japan's historical connections with the rest of Asia, as evidenced by the establishment of museums and cultural institutions that focused on Japan's interactions with other Asian countries \cite{Carlile2010ExploringA}. This growing awareness of Japan's place within Asia contributed to the development of a more ``Asian'' national identity in the 21st century.



\bibliographystyle{unsrtnat}
\bibliography{references}

%\appendix
%\section{Parameters}

\end{document}
